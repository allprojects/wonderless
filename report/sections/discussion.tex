% !TEX root = ../main.tex

\section{Discussion}
\label{discussion}

\subsection{Identifying Serverless applications} \label{discOnSF}
The starting point for constructing the Wonderless is identifying Serverless 
applications. We discuss two alternatives to our approach.

First, one can use services like GHTorrent~\cite{gousios2012ghtorrent}, 
an offline mirror of the GitHub public event timeline, to search for repositories 
that contain specific keywords in their descriptions, topics, commits or other 
related attributes. Yet, the presence of a keyword does not guarantee 
that an application is Serverless. Analogously, an application may be Serverless 
without containing a specific keyword in the repository attributes.
Our approach is more precise in these regards, 
but it is limited to a specific framework. 
\emph{Serverless Framework} is widely used with more than 36 K GitHub stars 
and 15 M downloads. According to a review of Serverless 
frameworks~\cite{kritikos2018review}, \emph{Serverless Framework},
with the highest number of supported providers 
and programming languages, and with the deployment, testing, 
monitoring, and security offerings is the most comprehensive existing framework. 

Second, one can search for a configuration file that is 
specific to Serverless applications in the GitHub API. Unfortunately, to consider several vendors, 
one needs to find specific configuration files for each provider.
Worse, the Serverless configuration files are hardly distinguishable from other cloud setups. 
For example, one of the default Serverless configuration files in Amazon 
Web Services~(AWS) is $index.js$. On January 28, 2021, there were 
more than 202 M files with this name on GitHub, most not 
related to Serverless applications. 
In our approach, instead, we rely on a configuration file that is exclusively 
for Serverless applications, and has the same default name across 
different providers and platforms. \emph{Serverless Framework} 
satisfies both these criteria.

\subsection{Use cases for Wonderless}
We envision using Wonderless in future studies 
concerning different directions.
% as a data source for characteristic recognition, 
% trend identification, evolution investigation, and performance evaluation.

First, Wonderless can be used to study several aspects that so far have not
been considered or have been only marginally touched by researchers.
%including security of serverless applications and continuous integration.
For example, to the best of our knowledge, there is no comprehensive 
empirical analysis of security of Serverless applications. 
As a starting point, Hong at al.~\cite{216833} proposed a catalogue of
security patterns for Serverless computing which could be used to assess
the security design of the applications in Wonderless.

%\TODO{say that to the best of our knowledge there is nothing about security.
%Is it true?}
%\NEW{Almost no. For example:
%\begin{itemize}
%	\item \url{https://www.usenix.org/system/files/conference/hotcloud18/hotcloud18-paper-hong.pdf}
%	\item \url{https://ieeexplore.ieee.org/stamp/stamp.jsp?arnumber=9180214}
%	\item \url{https://arxiv.org/pdf/1802.08984.pdf}
%	\item \url{https://link.springer.com/content/pdf/10.1007/s00450-019-00413-w.pdf}
%\end{itemize}
%}
%
Recently, Rahman et al.~\cite{DBLP:journals/ese/RahmanFW20} developed a 
catalogue of antipatterns for Infrastructure as Code. We believe that an 
analogous study of antipatterns for Serverless computing would be beneficial.
Similarly, Obetz et al. have proposed a novel static analysis technique that is  
specific to Serverless computing and they demonstrated it on 
seven Serverless applications~\cite{10.5555/3357034.3357059}.

Second, as Wonderless includes the full history of each Serverless application, 
it provides developers the opportunity to study how Serverless computing 
projects evolve over time~\cite{sousa2020characterizing, du2020understanding, wen2020empirical}. 
It would be interesting to compare the results with analogous studies in Wonderless.

Finally, Wonderless, with 1,877 data points, can be used to extend existing 
studies along several dimensions. Concerning number of applications,
studies that have focused on datasets in the order of tens of 
applications~\cite{eismann2020serverless} can be extended to a much larger scale.
Along the methodology axes, Wonderless can be used to complement the knowledge collected 
with surveys, and developer interviews with first-hand analysis of
Serverless applications~\cite{leitner2019mixed}.
Along the technology axes, since Wonderless does not focus on a specific vendor, it can be used
to discover whether existing analyses carried out for one vendor
generalize to others~\cite{spillner2019quantitative}.











