% !TEX root = ../main.tex

\section{Discussion}
\label{discussion}

\paragraph{identifying Serverless applications}
The starting point for constructing the Wonderless is identifying Serverless 
applications. We discuss two alternatives to our approach.

First, one use services like GHTorrent\cite{gousios2012ghtorrent}, 
an offline mirror of the GitHub public event timeline, to search for repositories 
that contain specific keywords in their descriptions, topics, commits or other 
related attributes. However, the presence of a keyword does not guarantee 
that an application is Serverless. Analogously, an application may be Serverless 
without containing a specific keyword in the repository attributes.

Second, one can use the GitHub API to search for a configuration file that is 
specific to Serverless applications. Unfortunately, to consider several vendors. 
one needs to find specific configuration files for each provider.
Worse, the serverless configuration files are hardly distinguishable from other cloud setups. 
For example, one of the default Serverless configuration files in Amazon 
Web Services\,(AWS) is $index.js$. To date\footnote{\today}, 
more than $135 \, M$ files with this name are on GitHub, most not 
related to Serverless applications. 
% This procedure can lead to a massive dataset with a large number 
% of unrelated data points.

In our approach, instead, we rely on a configuration file that is exclusively 
for Serverless applications, and has the same default name across 
different cloud providers and Serverless platforms. 
$Serverless \; Framework$\footnote{\url{https://www.serverless.com}} 
with more than 36\,K GitHub stars and 15\,M downloads satisfies both these 
criteria. According to a review of Serverless frameworks\cite{kritikos2018review}, 
the $Serverless \; Framework$ with the highest number of supported providers 
and programming languages, and with the advance deployment, testing, 
monitoring, and security offerings is the most comprehensive existing framework. 









