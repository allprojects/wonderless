% !TEX root = ../main.tex
\section{Discussion}
\label{discussion}
The starting point for constructing the Wonderless is identifying Serverless 
applications. One way to achieve this end is to use services like 
GHTorrent\cite{gousios2012ghtorrent}, 
an offline mirror of the GitHub public event timeline, to search for repositories 
that contain specific keywords in their descriptions, topics, commits or other 
related attributes. The problem with this method is that containing a keyword 
does not necessarily guarantee that an application is Serverless. Besides, an 
application may be Serverless without containing a specific keyword in the 
attributes of its related repository.

The other way is to use GitHub API to search for a configuration file that is 
particularly for Serverless applications. The documentation of popular cloud 
providers reveals that the Serverless configuration file has the same default 
name as the configuration file for other cloud offerings. 
For example, one of the default Serverless configuration files in Amazon 
Web Services\,(AWS) is $index.js$. To this date\footnote{\today}, there 
are more than $135 \, M$ files with this name in GitHub, and they are not 
necessarily related to a Serverless application. To make matters worse, 
searching for this file will limit the dataset to only the applications supported 
by AWS. To include the Serverless applications provided by other vendors, 
one requires to search for the specific configuration files of those providers 
individually. This procedure can lead to a massive dataset with a large number 
of unrelated data points.

These challenges point to search for a configuration file that is first exclusively 
for Serverless applications, and second, it has the same default name across 
different cloud providers and Serverless platforms. 
$Serverless \; Framework$\footnote{\url{https://www.serverless.com}} 
with more than 36\,K GitHub stars and 15\,M downloads satisfies both these 
criteria. According to a review of Serverless frameworks\cite{kritikos2018review}, 
the $Serverless \; Framework$ with the highest number of supported providers 
and programming languages, and with the advance deployment, testing, 
monitoring, and security offerings is the most comprehensive existing framework. 