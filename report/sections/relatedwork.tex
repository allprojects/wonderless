\section{Related Work}
\label{relatedwork}

\subsection{Related datasets}
Eismann et al. in \cite{eismann2020serverless} analyze a data source 
containing $32$ open-source projects. These projects are a subset of 
an existing dataset containing $2,194$ repositories introduced by 
Pavlov et al.\cite{pavlov2019serverless}. The initial dataset was extracted 
from GitHub using GHTorrent based on keyword search and repositories' 
creation date. Eismann et al. applied several additional filters to this dataset 
to take out active and real-world Serverless applications resulting in only $32$
projects.These filters are based on the number of files, commits, 
contributors, and watchers of the projects, along with the manual walks 
through the repositories. Wonderless extends and complements 
this dataset by adding a large number of Serverless applications that 
have been collected fully automatically.
AWS SAR enables developers to store and share reusable applications. 
This repository can also be used as a dataset to investigate Serverless 
applications. However, it is bounded to AWS by only containing the 
applications provided by AWS and the applications developed by 
AWS verified authors.

\subsection{Related use cases}
With the increasing popularity of \faas platforms and Serverless applications, 
we are observing more and more studies on a broad range of topics in 
Serverless computing from both academia and industry. Wonderless can 
be used in most of these studies as a data source for characteristic recognition, 
trend identification, evolution investigation, and performance evaluation.

\emph{Recognize Serverless characteristics.}
Eismann et al.\cite{eismann2020serverless} attempt to address the following 
three main questions to clarify the current picture of Serverless computing: 
why do so many companies adopt serverless?, when are serverless applications 
well suited?, and how are serverless applications currently implemented?. For this 
purpose, they analyze 89 Serverless applications from open-source projects, 
industrial sources, academic literature, and scientific computing. We discuss 
the data source for this study in section\,\ref{relatedwork}. Wonderless with 
$1,913$ data points can be used to complement such studies on a more 
comprehensive and larger scale.

\emph{Identify trends in Serverless development.}
Leitner et al.\cite{leitner2019mixed} present the results of a mixed-method 
study consisting of interviews, a literature review, and a Web-based survey 
to identify best practices for building Serverless applications. 
They collect the data based on the programmer's experience while 
developing the application. Wonderless can be used to validate the 
results of this study from a technical point of view and in the code-level.

\emph{Investigate the evolution of Serverless platforms.}
Spillner\cite{spillner2019quantitative} investigates the high-level 
implementation statistics and evolution of Lambda functions through the 
Amazon Web Services Serverless Application Repository\,
(AWS SAR)\footnote{\url{https://aws.amazon.com/serverless/serverlessrepo/}}. 
Wonderless with the feature of having real-world applications from
providers other than AWS, enables researchers to replicate a similar 
study related to those providers and platforms.

\emph{Serverless cost and performance evaluations.} 
Several studies\cite{wang2018peeking, 
	lloyd2018serverless, figiela2018performance, lee2018evaluation, 
	mcgrath2017serverless, back2018using, mohanty2018evaluation} 
have evaluated the performance of popular \faas solutions by running 
benchmark functions across different platforms. The metrics for these 
evaluations cover a wide range, including computing, memory, network, and 
I/O parameters. Another class of studies investigates the cost of \faas platforms 
based on different utilization metrics. Work in \cite{adzic2017serverless} 
presents how adopting Lambda deployment architecture reduces hosting costs 
by studying the migration of two industrial cases of early adopters.
Jackson and Clynch\cite{jackson2018investigation} study the impact of 
language runtime on the cost of Serverless functions in AWS Lambda and 
Azure Functions. In \cite{bortolini2019investigating}, Bortolini and Obelheiro 
investigate the cost variations within and across \faas platforms 
based on the choice of memory allocation, \faas provider, and 
programming language. Wonderless with the feature of spanning across 
all the popular vendors and platforms, provides valid case studies for 
these classes of studies.



