% !TEX root = ../main.tex

\section{Related Work}
\label{relatedwork}

\subsection{Related datasets}
Eismann et al. in \cite{eismann2020serverless} analyze
$32$ open-source projects. These projects are a subset of 
an existing dataset containing $2,194$ repositories introduced by 
Pavlov et al.\cite{pavlov2019serverless}. The initial dataset was extracted 
from GitHub using GHTorrent based on keyword search and repositories' 
creation date. Eismann et al. applied several additional filters to this dataset 
to take out active and real-world Serverless applications resulting in only $32$
projects. These filters are based on the number of files, commits, 
contributors, and watchers of the projects, along with the manual walks 
through the repositories. Wonderless extends and complements 
this dataset by adding a large number of Serverless applications that 
have been collected fully automatically.

The Amazon Web Services Serverless Application Repository
(AWS SAR)\footnote{\url{https://aws.amazon.com/serverless/serverlessrepo/}} 
enables developers to store and share reusable applications. 
This repository can also be used as a dataset to investigate Serverless 
applications. Yet, it is limited to AWS: it only contains 
applications provided by AWS and applications developed by 
AWS verified authors.


\subsection{Studies on Serverless computing}

With the increasing popularity of \faas platforms and Serverless applications, 
more and more studies have been conducted on a broad range of topics in 
Serverless computing. 

%\emph{Recognize Serverless characteristics.}
Eismann et al.~\cite{eismann2020serverless} provide a  
high-level picture of Serverless computing, including 
company adoption, suitable application context, and implementation 
of serverless applications.
In the study, the authors analyze 89 Serverless applications from open-source projects, 
industrial sources, academic literature, and scientific computing. 


Leitner et al.~\cite{leitner2019mixed} present the results of a mixed-method 
study consisting of interviews, a literature review, and a Web-based survey 
to identify best practices for building Serverless applications. 
The authors collect the data based on the programmer's experience while 
developing the application. 

%\emph{Investigate the evolution of Serverless platforms.}
Spillner~\cite{spillner2019quantitative} investigates the 
evolution of Lambda functions through AWS SAR. 
In this study, the evolution of function-level metadata, 
code-level metadata, and code-level implementation of 
Lambda functions is investigated by continuous observation, 
extraction, mining and conflation of repositories in AWS SAR.


Several studies~\cite{wang2018peeking,lloyd2018serverless,figiela2018performance,lee2018evaluation,mcgrath2017serverless,back2018using,mohanty2018evaluation} 
have evaluated the performance of popular \faas solutions by running 
benchmark functions across different platforms. The metrics include time, 
memory, network, and I/O. Another class of studies investigates the cost of \faas platforms 
based on different utilization metrics. Lenarduzzi and Panichella~\cite{adzic2017serverless} 
study the migration of two industrial cases of early adopters
and show how Lambda deployment architectures reduces hosting costs. 
Jackson and Clynch~\cite{jackson2018investigation} study the impact of 
language runtime on the cost of Serverless functions in AWS Lambda and 
Azure Functions. Bortolini and Obelheiro~\cite{bortolini2019investigating} 
investigate the cost variations within and across \faas platforms 
based on memory allocation, \faas provider, and 
programming language. 
% Wonderless with the feature of spanning across 
% all the popular vendors and platforms, provides valid case studies for 
% these classes of studies.



