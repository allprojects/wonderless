% !TEX root = ../main.tex
\section{threats to Validity}
\label{limitations}

%To create an extensive and representative dataset for future research 
%on Serverless development, we collected 1,913 Serverless applications. 
%However, in the process, we made a few choices that can impact the dataset. 
%

First, we limited the dataset to public projects available in GitHub. As of December 2020, 
GitHub with more than 64\,M\,\footnote{\url{https://github.com/search}} 
developers and 36\,M\,\footnote{\url{https://github.com/search?q=is:public}} public 
repositories is the home for the largest open-source community in the world. 
We believe this community is comprehensively representative of how Serverless 
computing is being adopted in practice.
%

Second, Wonderless is restricted to the applications 
that are developed using the \emph{Serverless Framework}. 
However, not every developer uses a framework to develop a Serverless application. 
They may directly use offerings of a provider, or they may use self-hosted solutions. 
%
As thoroughly 
described in section\,\ref{phaseA}, \emph{Serverless Framework} is the 
most popular open-source solution for developing Serverless applications 
based on the GitHub stars and the results of a prior study~\cite{kritikos2018review}. 
We believe that this issue does not significantly impact the characteristics of a 
Serverless application and consequently, generalization of the dataset.

Another issue is that the filtering procedure may not have removed all 
uninteresting cases, including toy software and stub applications.
To this end, we randomly selected 1\% of the projects, 
and we manually checked the nature of them. 
Only one of the projects was using a template from Serverless framework, 
and other projects were active Serverless applications








