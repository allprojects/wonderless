% !TEX root = ../main.tex
\section{threats to Validity}
\label{limitations}

%To create an extensive and representative dataset for future research 
%on Serverless development, we collected 1,913 Serverless applications. 
%However, in the process, we made a few choices that can impact the dataset. 
%

First, we limited the dataset to public projects available in GitHub. As of December 2020, 
GitHub with more than 64\,M\,\footnote{\url{https://github.com/search}} 
developers and 36\,M\,\footnote{\url{https://github.com/search?q=is:public}} public 
repositories, is home for the largest open-source communities. 
We believe that such communities are 
% comprehensively 
representative of 
how Serverless computing is adopted in practice.
%

Second, Wonderless is restricted to the applications 
developed with the \emph{Serverless Framework}. 
However, not every developer uses a framework to program Serverless applications. 
They may directly use a provider, or they may adopt self-hosted solutions. 
%
As discussed in section\,\ref{phaseA}, the \emph{Serverless Framework} is the 
most popular open-source solution for Serverless applications -- 
based on GitHub stars and the results of a prior study~\cite{kritikos2018review}. 
Hence, we believe that the issue above does not significantly diminish 
% the characteristics of a Serverless application and consequently, 
the possibility of generalizing the results obtained from the dataset.

Another issue is that the filtering procedure may not have removed all 
uninteresting cases, including toy software and stub applications.
To this end, we randomly selected 1\% of the projects, 
and we manually checked them. 
Only one of the projects was using a template from Serverless framework, 
and other projects were active Serverless applications.








