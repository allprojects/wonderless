% !TEX root = ../main.tex

\section{Introduction}

Since the advent of the \faas model for Serverless computing in 2014  
supported by Amazon,
% \footnote{\url{https://aws.amazon.com/lambda/}}, 
all the major cloud service providers, including 
Google,
% \footnote{\url{https://cloud.google.com/functions}}, 
Microsoft,
%\footnote{\url{https://azure.microsoft.com/services/functions/}}, 
and IBM,
%\footnote{\url{https://cloud.ibm.com/functions/}}
have introduced equivalent services. 
%
In addition to these offers, a growing number of open-source platforms 
such as Apache OpenWhisk,
%\footnote{\url{https://openwhisk.apache.org}}, 
OpenFaas,
%\footnote{ \url{https://www.openfaas.com}}, 
and Kubeless
%\footnote{\url{https://kubeless.io}} 
supporting the \faas programming model are being actively developed and maintained.

In contrast to the traditional cloud offerings where users explicitly provision or configure 
backend services, in \faas, infrastructure management is left to the provider.
Thanks to this approach developers focus on the application logic and 
are not concerned about the infrastructure. 
Programmers develop a Serverless application uploading the code of one or more 
functions to the cloud and selecting the trigger events~(e.g., a REST request, 
a file upload) that activate the function. The cloud provider is then responsible for 
deployment and resource provisioning. As a consequence, the developers are  
charged only for the resources that the application actively requires -- eliminating the 
need for worst case scenario over-provisioning.

Even though the \faas model simplifies programmers' tasks 
% by providing an environment to focus on developing applications in high-level languages, 
as discussed before, it also introduces several challenges.
 %for the development and performance of such applications. 

First, developers are forced to adopt a programming model that, in practice,
significantly departs from the well known imperative and functional paradigms.
% A number of aspects contribute to that.
For example, similar to functional programming, in the Serverless model, 
programs are required to be stateless~\cite{hellerstein2018serverless} 
to enable autoscaling via automated function parallelization.
On the other hand, function composition, which is the cornerstone of functional programming
is often considered an antipattern in Serverless computing~\cite{baldini2017serverless}.
%
As a result, developers resort to a programming model that 
resembles the imperative paradigm and routines.
Yet, there are fundamental differences, including the fact that 
the state of a function is not preserved across several executions and
different functions that belong to the same application may not even execute 
on the same machine. As a result, it is a common solution 
to use external shared storage systems 
to save intermediate data across functions 
executions~\cite{klimovic2018understanding}.


Second, the performance of Serverless applications is much harder to 
predict than traditional cloud applications. A number of aspects concur 
to this state of things.
One is a consequence of the common programming practice
of adopting an external storage systems: cross-function communication 
is slower and costlier than point-to-point networking.
Other issues are inherently due to the characteristics of Serverless systems,
including lack of information on data locality~\cite{DBLP:journals/corr/abs-1902-03383},
delays due to containers startup time~(cold start times)~\cite{manner2018cold}, 
complex triggering processes~\cite{pelle2019towards}, and
limited lifetime of functions~\cite{hellerstein2018serverless}.

Third, we lack tool support for various aspects of
Serverless software development, including testing, debugging 
and continuous integration~\cite{lenarduzzi2020serverless, nupponen2020serverless}.
Like a small monolithic system, a function can be unit tested 
locally before deployment. 
However, system-level and integration testing and debugging can 
be more complicated when more than a single function is involved 
in the application. In a Serverless application with several functions, 
most environmental dependencies are only available at runtime, 
making local integration testing and debugging impossible in 
some cases~\cite{leitner2019mixed}.


The issues above, exacerbated by the vendor lock-in that is currently 
characterizing vast amount of the Serverless computing market, 
pose the major challenges that slower the adoption of the \faas model. 
A first step to address the challenges above is to achieve a 
better understanding of how the Serverless model is used in practice. 
Unfortunately, relatively little is known about the characteristics 
or behavior of real-world Serverless applications. Existing studies focus 
on specific aspects, such as evolution~\cite{spillner2019quantitative} 
or performance\cite{wang2018peeking,lloyd2018serverless},
but they do not provide a general-purpose dataset for Serverless computing.
Other researchers either applied different research methodologies, 
such as developers interviews and literature surveys~\cite{leitner2019mixed},
or target a some tens of applications~\cite{eismann2020serverless}.


In this work, we bridge this gap by providing a dataset of real-world Serverless 
applications which is ready to use for researchers who are interested in investigating
Faas applications. Our dataset, `\emph{Wonderless}', is publicly 
available\,\footnote{\url{https://github.com}} 
and the source code\,\footnote{\url{https://github.com}} 
is open for replication as well as for extension.











