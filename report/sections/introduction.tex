% !TEX root = ../main.tex

\section{Introduction}

Since the advent of Serverless computing with an event-based \faas model in 2014 by 
Amazon,
% \footnote{\url{https://aws.amazon.com/lambda/}}, 
all the major cloud service providers including 
Google,
% \footnote{\url{https://cloud.google.com/functions}}, 
Microsoft,
%\footnote{\url{https://azure.microsoft.com/services/functions/}}, 
and IBM
%\footnote{\url{https://cloud.ibm.com/functions/}}
introduced equivalent services. 
%
In addition to these offers, a growing number of open-source platforms 
such as Apache OpenWhisk,
%\footnote{\url{https://openwhisk.apache.org}}, 
OpenFaas,
%\footnote{ \url{https://www.openfaas.com}}, 
and Kubeless
%\footnote{\url{https://kubeless.io}} 
supporting the \faas programming model are being actively developed and maintained.

In contrast to the traditional cloud, where users explicitly provision or configure 
backend services, in \faas infrastructure management is left to the provider.
With this approach enables focus on the functionalities of their programs 
and do not need to be concerned about infrastructure. 
Users can develop a Serverless application by uploading the code of their 
functions to the cloud provides and by selecting trigger events~(e.g., a REST request, 
a file upload) that activate the application. The cloud provider is then responsible for the deployment 
and for resource provisioning. As a consequence, the user only has 
to pay for the active resources that the application requires -- eliminating the 
need for over-provisioning to be ready for peak load.

Even though the \faas model simplifies programmers' tasks by providing 
an environment to focus on developing applications in high-level 
languages, it also introduces several challenges.
 %for the development and performance of such applications. 

First, the programming model adopted by developers 
significantly departs from the well known imperative and functional paradigms.
A number of constraints contribute to that.
Similar to functional programming, in the Serverless model, 
programs are required to be stateless~\cite{hellerstein2018serverless} to enable autoscaling.
However, function composition, which is the cornerstone of functional programming
is often considered an antipattern~\cite{TODO}.
%
As a result, developers resort to a programming model that 
resembles the imperative paradigm and routines --
for example using external shared 
storage systems to save intermediate data~\cite{klimovic2018understanding}.
Yet, there are fundamental differences, including the fact that 
the state of a function is not preserved through the executions ands and
different functions that belong to the same application may not even execute 
on the same machine.

Second, the performance of Serverless applications is much harder to 
predict that traditional VM cloud applications. A number of aspects concur to that.
Some are a consequence of programming practives:
using the external storage systems make the 
communication slower and costlier than point-to-point networking.
Others are inherently due to the characteristics of Serverless systems
including 
lack of information on data locality~\cite{DBLP:journals/corr/abs-1902-03383},
delays caused by the time needed 
for the container starts~(cold start times)~\cite{manner2018cold}, 
complex triggering processes~\cite{pelle2019towards}, and
limited lifetime of functions~\cite{hellerstein2018serverless}, 

Third, there is a lack of tools for covering various aspects of
Serverless software development, including testing, debugging 
and continuous integration. TODO: any more evidence to cite?
	
The issues above, exacerbated by the vendor lock-in that is currently 
characterizing vast amount of the Serverless computing market, 
pose a the major challenge that slower the adoption of the \faas model. 

Unfortunatly, relatively little is known about the characteristics 
or behavior of real-world Serverless application. Existing studies focus 
on specific aspects, such as evolution~\cite{spillner2019quantitative} 
or performance\cite{wang2018peeking,lloyd2018serverless}.
But do not provide a general-purpose dataset for Serverless computing.
Other researchers either applied different research methodologies, 
such as developers interviews and literature surveys~\cite{leitner2019mixed},
or target a some tens of applications~\cite{eismann2020serverless}.
A first step to address the challenges above is to achieve a 
better understanding of how the Serverless model is used in practice. 


In this work, we bridge this gap by providing a dataset of real-world Serverless 
applications which is ready to use for researchers who are interested in investigating
Faas applications. Our dataset, `\emph{Wonderless}', is publicly 
available\,\footnote{\url{https://github.com}} 
and the source code\,\footnote{\url{https://github.com}} 
is open for replication as well as for extension.











