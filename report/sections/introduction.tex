\section{Introduction}

Since the arrival of Serverless computing with an event-based \faas model in 2014 by 
Amazon,
% \footnote{\url{https://aws.amazon.com/lambda/}}, 
all the major cloud service providers including 
Google,
% \footnote{\url{https://cloud.google.com/functions}}, 
Microsoft,
%\footnote{\url{https://azure.microsoft.com/services/functions/}}, 
and IBM
%\footnote{\url{https://cloud.ibm.com/functions/}}
are offering equivalent services. 
%
In addition to these offerings, a growing number of open-source platforms 
such as Apache OpenWhisk,
%\footnote{\url{https://openwhisk.apache.org}}, 
OpenFaas,
%\footnote{ \url{https://www.openfaas.com}}, 
and Kubeless
%\footnote{\url{https://kubeless.io}} 
with the \faas programming model are being actively developed and maintained.

In contrast to traditional cloud offerings that users explicitly provision or configure 
backend services, in \faas, the infrastructure becomes the provider's burden.
This enables users to concentrate on the functionality of their programs 
without having to be concerned with maintenance issues. 
Users can develop a Serverless application only by uploading the code of their 
functions to the cloud and selecting trigger events~(e.g., a REST request, 
a file upload). The provider is then responsible for the deployment of the 
application and provisioning resources. Consequently, the user only has 
to pay for the active resources that the application uses, eliminating the 
need to pay for idle components of the application.

Even though the current \faas model simplifies users' tasks by providing 
them an environment to focus on developing applications in high-level 
languages, it also proposes several challenges on the development 
and performance of such applications. One of the main challenges 
that dramatically reduces the overall performance and utility of 
Serverless applications is that programs require to be 
stateless\cite{hellerstein2018serverless}.
Due to the autoscaling nature of Serverless applications, the state of 
a function is not preserved through the execution. This along with the
possibility that different functions of an application may not execute 
on the same machine force developers to use external shared 
storage systems for exchanging intermediate data\cite{klimovic2018understanding}.
Needless to say that using the external storage systems make the 
communication slower and costlier than point-to-point networking.

This limitation together with the delays caused by the time needed 
for the container starts~(cold start times)\cite{manner2018cold} or 
complex triggering processes\cite{pelle2019towards}, 
limited lifetime of functions\cite{hellerstein2018serverless}, 
vendor lock-in, and lack of tools for testing and debugging are among 
the major challenges that slower the adoption of the \faas model. 

The primary requirement to alleviate these challenges is having a 
deep understanding of Serverless characteristics. However, relatively little is 
known about the characteristics or behavior of a Serverless application. 
We believe that this knowledge can be achieved by investigating the use 
cases of the \faas model in the real-world.

In this study, we attempt to fulfill this urgent need of researchers for 
Serverless data sources by providing a dataset of real-world Serverless 
applications in GitHub, the home of the largest open-source community. 
Our dataset, `\emph{Wonderless}', is publicly 
available\,\footnote{\url{https://github.com}} 
and the source code\,\footnote{\url{https://github.com}} 
is open for replication as well as the extension.
